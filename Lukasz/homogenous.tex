\documentclass[11pt]{article}
\usepackage{amssymb}
\usepackage{amsmath}
\usepackage{color}
%\usepackage{lscape}
\usepackage{graphicx}
%\usepackage{caption2}
%\usepackage{epic,eepicemu}
%\input{epsfig}
%\usepackage{endnotes}

\renewcommand{\baselinestretch}{1.5}
%\pagestyle{plain}
%\pagestyle{empty}
\topmargin=0cm
\oddsidemargin=0cm
%\evensidemargin=-cm
\textwidth=16cm \textheight=21cm
%\parindent=0.0cm
\newcommand{\tdc}[1]{\textcolor{red}{\textsc{#1}}}
\newtheorem{lemma}{Lemma}
\newtheorem{prop}{Proposition}

\begin{document}
\section*{Exercize 1: Identification of conduct}

\begin{itemize}
\item The task is to investigate competition in Dutch coffee market.
\item Time-series monthly data contains information about the Dutch coffee
market during the period 1990-1996 (more information in Bettendorf and Verboven (1998)).
\item The data include the following variables:
\begin{itemize}
\item month: year and month of observation;
\item qu: per capita consumption of roasted coffee in kg;
\item cprice: price of roasted coffee per kg in current guilders;
\item tprice: price of per kg tea in current guilders;
\item oprice: price index for other goods;
\item incom: income per capita in current quilders;
\item q1-q4: dummy variables for seasons 1 to 4;
\item bprice: price of coffee beans per kg in current guilders;
\item wprice: price of labor per man hours (work 160 hours per month).
\end{itemize}
\end{itemize}

\section*{Empirical model}

\begin{itemize}
\item Market demand is assumed to be linear:
\begin{equation}
Q_t=\beta (\alpha - P_t)^\gamma+\epsilon_t
\end{equation}
which assuming that $\alpha=0$ and $\gamma<0$ is a log-linear demand function:
\begin{equation}
ln(Q_t)=ln(-\beta)+\gamma ln(P_t)+\epsilon_t
\end{equation}
where $Q_t$ is total output in the market and $P_t$ is the market price and $\epsilon_t$ is an
error term, $\epsilon_t\sim N(0,\sigma^{2})$
\item The coffee market is characterized by a relatively simple production technology
with constant marginal cost:
\begin{equation}
c=c_0+kP_{coffe beans},
\end{equation}
where $c_0$ represents all variable costs other than those related to coffee beans,
i.e., labor and packages; and k is a parameter that measures the fixed technology in production.
It is estimated that
one kg of roasted coffee requires 1.19 kg of beans. The $c_0$ is estimated to be
around 4 guilders.
\item The profit for firm i is given by:
\begin{equation}
\pi(q_i,q_{-i})=(P(Q)-c)q_i
\end{equation}
where $Q=\sum_J{q_i}$. The first order condition of profit maximization
in the Cournot model implies:
\begin{equation}
\frac{\partial\pi(q_i,q_{-i})}{\partial q_i}=0\Rightarrow P(Q)+\lambda_i q_i\frac{\partial P(Q)}{\partial Q}=c
\end{equation}
where $\lambda_i=1+\sum_{j\neq i}\frac{\partial q_j}{\partial q_i}$ represents conjectural variation.
\begin{equation}
P(Q)+\lambda_i \frac{q_i}{Q}Q\frac{\partial P(Q)}{\partial Q}=c
\end{equation}
In the case of N identical firms we have: $\frac{q_i}{Q}=\frac{1}{N}$:
\begin{equation}
P(Q)+\theta Q\frac{\partial P(Q)}{\partial Q}=c
\end{equation}
\item The conduct parameter can take the following values:
\begin{itemize}
\item $\theta=0$ for perfect competition
\item $0<\theta<1$ for oligopoly
\item $\theta=1$ for for monopoly or collusion
\end{itemize}
\item If we have access to information
about costs, conduct parameter $\theta$ can be expressed in the following way:
\begin{equation}
\theta=-\gamma\frac{P-c}{P}\equiv L_{\eta}
\end{equation}
where $\eta (P)$ is the elasticity of demand and $L_{\eta}$ is the adjusted Lerner index,
i.e., Lerner index adjusted for elasticity (Genesove and Mullin, 1998).
\item The market price can be written as a function of the conduct parameter $\theta$, the
estimated demand, and cost parameters:
\begin{equation}
P(c)=\frac{\gamma}{\gamma+\theta}c
\end{equation}
where $\gamma$ is the estimated demand elasticity in log-linear demand specification.
\end{itemize}

\section*{Your task}
\begin{itemize}
\item Analyze data by computing simple statistics and graphical illustration.
\item Estimate demand for roasted coffee using reasonable explanatory variables and instrumental variables.
\item Explain what allows for identification of conduct in this model.
\item Estimate Lerner index adjusted for elasticity and conduct parameter, and provide interpretation.
\end{itemize}

\end{document} 